
\documentclass[parskip]{cs4rep}

\begin{document}

\title{An LLVM based compiler from Objective-C to Dalvik Virtual Machine}

\author{Stanislav Manilov}

% to choose your degree
% please un-comment just one of the following
%\degree{Artificial Intelligence and Computer Science}
%\degree{Artificial Intelligence and Software Engineering}
%\degree{Artificial Intelligence and Mathematics}
%\degree{Artificial Intelligence and Psychology }   
%\degree{Artificial Intelligence with Psychology }   
%\degree{Linguistics and Artificial Intelligence}    
%\degree{Computer Science}
%\degree{Software Engineering}
%\degree{Computer Science and Electronics}    
%\degree{Electronics and Software Engineering}    
%\degree{Computer Science and Management Science}    
\degree{Computer Science and Mathematics}
%\degree{Computer Science and Physics}  
%\degree{Computer Science and Statistics}    

% to choose your report type
% please un-comment just one of the following
%\project{Undergraduate Dissertation} % CS&E, E&SE, AI&L
%\project{Undergraduate Thesis} % AI%Psy
\project{4th Year Project Report}

\date{\today}

\abstract{
This document gives an example of Informatics Project Report style
(previously {\tt cs4rep} style).
The file {\tt skeleton.tex} generates this document and can be 
used to get a ``skeleton'' for your thesis.
The abstract should summarise your report and fit in the space on the 
first page.
%
You may, of course, use any other software to write your report,
as long as you follow the same style. That means: producing a title
page as given here, 
% policy changed as of November 24 1999 (in TC)
%printing in double space, 
and including a table of
contents and bibliography.
}

\maketitle

\section*{Acknowledgements}
Acknowledgements go here. 
The class is a modification of the {\tt cs4rep} style 
used in the computer science department until 1998-9.

\tableofcontents

%\pagenumbering{arabic}


\chapter{Introduction}

The document structure should include:
\begin{itemize}
\item
The title page  in the format used above.
\item
An optional acknowledgements page.
\item
The table of contents.
\item
The report text divided into chapters as appropriate.
\item
The bibliography.
\end{itemize}

Commands for generating the title page appear in the skeleton file and
are self explanatory.
The file also includes commands to choose your report type (project
report, thesis or dissertation) and degree.
These will be placed in the appropriate place in the title page. 

The default behaviour of the class is to produce documents typeset in
12 point, 
and appropriate for doubled sided printing
(all new chapters appearing on the first clear right-hand page).
Regardless of the formatting system you use, 
it is recommended that you submit your thesis printed (or copied) 
double sided. 

{\bf NB} please note that the report should be printed single-spaced.
Previously advertised policy of printing in double space has changed
as of November 24th 1999 and is no longer valid.
Space recommendations are revised as follows: 
the dissertation should be around 40 sides in single space printing.
The page limit is 60 sides in single space printing.  Appendices are in
addition to the above and you should place detail here which may be too
much or not strictly necessary when reading the relevant section.

\section{Using Sections}

Divide your chapters in sub-parts as appropriate.

\section{Citations}

Note that citations 
(like \cite{P1} or \cite{P2})
can be generated using {\tt bibtex} or by
creating {\tt thebibliography} environment. This makes sure that the
table of contents includes an entry for the bibliography.
Of course you may use any other method as well.

\section{Class Options}

The only class option available is {\tt parskip}.
It alters the paragraph formatting so that each paragraph is separated by
a vertical space, and there is no indentation at the start of each
paragraph. 
This option is used in the current document.
See {\tt documentclass} in the skeleton file for usage. 

\section{Restrictions}

The class does not allow the use of {\tt listoffigures} or {\tt listoftables}.

\chapter{The Real Thing}

Of course
you may want to use several chapters and much more text than here.

% use the following and \cite{} as above if you use bibtex
% otherwise generate bibtem entries
\bibliographystyle{plain}
\bibliography{mybibfile}

\end{document}


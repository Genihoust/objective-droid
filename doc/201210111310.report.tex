\documentclass[a4paper,11pt]{article}
\author{Stanislav Manilov}
\title{Objective-Droid: Honours Project Progress Report 1}
\begin{document}
\maketitle
\section{Overview}
Objective-Droid is a compiler that compiles applications originally written
for iOS to Android. The essence of the project is writing a backend that
translates LLVM code to Dalvik bytecode.
\section{Motivation}
The problem was inspired by the issue of portability of mobile code.
Currently, there are tools to build multiplatform mobile applications, but
none to re-target, or port, applications that are already developed for a
specific platform.

In addition to being an interesting project, Objective-Droid can
potentially make: Objective-C more popular, Dalvik VM easier to target,
developers worry less about portability, and users have access to wider range
of apps. Everyone has a reason to be happy!
\section{Issues}
There are several potential problems that can arise for the project. 

First, iOS libraries are proprietary. There have been attempts to re-write them
in open-source code, notably GNUstep, but it is still version 0.x. Using it is
outside of the scope of the project, but possible.

Second, translating idioms for writing apps would not be a trivial task. This might
require some work on the front-end, which is outside the scope of this project as
well.

Third, and more importantly, during the first group meeting about the
honours projects, the potential risk of having internal representation
instructions that are hard to match to DEX instructions was identified. It was
suggested that these, and ways to work around these, should be investigated.

In order to minimise potential risks, the aim of the project is narrowed down
to the ability to compile command-line programs. Also, it is important to
sustain good code quality and produce a software that is maintainable and easy
to build on.
\section{Implementation}
Objective-Droid is based on the LLVM Compiler Infrastructure. It uses clang
for a front-end, it's own backend, smali for assembling, and apktool for
packaging. Objective-C is compiled to LLVM code by clang, the backend then
translates it to smali code, smali assembles the smali code to Dalvik
bytecode, and finally, apktool packages the .dex files to an .apk file.
\section{Evaluation}
Two aspects of the compiler are going to be tested: correctness and
performance of the resulting programs.

Correctness is going to be tested by compiling a set of programs using
Objective-Droid for a mobile app, and gcc for a desktop app. If the results are
always the same, then there is a good chance that Objective-Droid is
producing correct programs.

Performance is going to be measured by compiling a specific benchmark programs
written in both Objective-C and Java. The Obj-C versions will be compiled
with Objective-Droid and the Java versions will be compiled using the Android
SDK. After that a set of metrics (speed, memory usage, energy usage, possibly
others) is going to be measured for both programs and compared.
\section{Timeline}
By the end of October a big portion of the research on the DEX format and
writing an LLVM backend will be researched. By the end of November there will
be a first prototype of the compiler with limited functionality. By the end of
December the functionality will be broaden and evaluation will have been
started. By the end of January a big portion of the evaluation will be done
and documentation will have been started. By the end of February the compiler
will be in a finished or near-finished state. By the end of March the report
will have been written.
\end{document}
